Type III Remailer Protocol Specifications

Message Format

Overview 

Type III remailer messages are composed of two main parts, the
headers and the payload. For technical reasons the there exist in
all messages two headers, the main header and the secondary header
that have the same structure. Each of them is further composed of up
to 16 sub headers, that are addressed and encrypted to the
intermediate nodes (remailers). Different headers sub headers and the
payload are encrypted under different keys when the message is created
or travels through the network, so we shall proceed with explaining
how the full message is structured but starting with the smallest
building block. 

Definitions

- if B is a byte array, B[i:j] (j bytes) is sub array starting at 
  byte i with length j.
- R(n) (n bytes) Generates n random bytes.
- Z(n) (n bytes) Generates n zero bytes
- Len(M) (2 bytes) is the length of message M (* bytes)
- x|y (Len(x)+Len(y) bytes) denotes x concatenated with y.
- PAD(M,L) (L bytes) pads the message M (Len(M) <= L) to length L
  using zeroes.

- H(M) (20 bytes) is the SHA-1 hash of M (* bytes)

- RSA_OAEP_Encrypt(K,M) (128 bytes): the encryption of M (73
  bytes) using the public key K as described in PKCS\#1.
- RSA_OAEP_Decrypt(K,M) (73 bytes) Gives the decryption of the
  message M (128 bytes) under the private key corresponding to K.
- Encrypt(K,M) (Len(M) bytes) Rijndael in Counter mode encryption 
  of message M using key K. 
- Decrypt(K,M,i,j) (j bytes) Rijndael counter mode decryption 
  using the key material byte i to j. Len(M) = j.
- PRNG(K, n) (n bytes) Produces n bytes of pseudo-random numbers.
  PRNG(K, n) = Encrypt(K, Z(n))

The sub-header structure and address extensions.

The sub header contains all the information that a node needs to know
in order to check the integrity of a message and route it through the
Internet. The sub header contains the following fields: 

Sub-header fields:

Version Major: 1 byte
Version Minor: 1 byte
Shared Secret: 16 bytes
Digest:        16 bytes
Flags:         1 byte
Address Size:  2 bytes
Address Type:  2 bytes
Address:       [Address Size] bytes

* The Version is present to manage concurrent versions of the
protocol. If a packet is received with a version that is not supported
it should be discarded. Nodes must advertise what versions of the
protocol they support.
* The Shared Secret is the base secret that is used to generate all
other keys for the operations of the node on the packet. It must be
kept secret and discarded as soon as the packet has been processed. 
* The Digest contains an integrity check of the remaining header (128*15
bytes in total). It is calculated on the remailing sub-headers and the
junk that will be present at then end of headers when the node will
receive the message. 
* The Flags can be either set or unset. Different bits mean different
things: 
bit 0 - SWAP: Perform the swap operation
bit 1 - END:  Indicates the final node
bit 2 - RTRN: Indicates the final node of a SURB (END must also be set)
bit 3 - EXADDRS: Indicates that the next sub-header(s) is part of the
address. 
Each of the above have a specific significance when the node processes
the message. 
* The Address Size indicates the total size of the next address, while
the type indicates how it should be parsed. Addresses can fill the
whole space left in the sub-header and also use part of the next sub
header(s) (the EXADDRS flag must be set). If a node receives an address
type it does not recognize it should discard the message. Nodes should
advertise what address types they support.

The sub header fields are encrypted using RSA after having been padded
using OAEP (or should it be OAEP+ or Anonymous OAEP?), using a 1024
bit key which results in a size of 128 bytes. The OAEP padding allows
86 bytes to be used. In case an additional block is added to fit the
address it should be a multiple of 128 bytes and should have the
following structure:
 
Address Extension:

Magic number:     4 bytes
Number of blocks: 1 byte
Address Data:     Variable
Padding:          Variable

* The magic number is set to be four zero bytes.
* The number of blocks indicates how many additional blocks the address
extension is using. The total number of bytes actually used by the
address is indicated by the ``Address size'' filed of the sub-header.  
* Padding of zeroes is used to make the size of the Address Extension a
multiple of 128 bytes.  

The Address Extension corresponding to a particular sub-header is
encrypted using the Encrypt function with key=Hash(Shared Secret,
``ADDRESS EXTENSION SECRET KEY'') and appended to the RSA encrypted
sub-header.

We will formally refer to the sub header structure as:
SHS(V, SK, D, F, A)
And to the RSA-OAEP encrypted sub-header structure as:
ESHS(PK, SK, D, F, A)

The address structure

There are three predefined types of addresses:

0x0001 TCP/IP4 (IP: 4 bytes, PORT: 2 bytes, KEYID: 16 bytes): 22 bytes
0x0002 SMTP   (EMAIL ADDRESS: Variable): Variable bytes
0x0003 RTRN   (NUM: 2 bytes, MASTER SECRET: 16 bytes): 18 bytes

The address type RTRN should be used in conjunction with stateless
reply blocks, to remind the recipient the master secret used to
derive the keys distributed to all the intermediate nodes, and their
number.

Replay Avoidance

The nodes MUST implement mechanism to make sure that messages or part
of messages cannot be replayed. To do this a hash of the secret
contained in the sub header is kept for as long as the public key
under which it is encrypted is in use. The Hash to be kept is computed
in the following way:

X = H(SharedSecret, ``REPLAY PREVENTION'')

The value X is not secret and should not be relied upon to be kept
secret. The integrity of the list should be secured and the X values
lists may be made public.

The header structure:

Two headers are present in a type III message: the main header and the
secondary header, that are swap at the crossover point. They both have
the same structure. 

A header is 16*128 bytes long and contains up to 16
sub-headers. Assuming that we have N sub headers SH0..SHN containing
secrets SK0..SKN, the header is constructed by appending headers
SH0..SHN together and with some random padding to achieve a total size
of 126*16 bytes. Then each sub header key is used to create a key
Hash(SharedSecret, ``HEADER SECRET KEY'') with which the part of the
header after the sub header (but including its address extension) is
encrypted with using the stream cipher. 
In practice headers and sub headers are constructed in parallel since
the Digest contained in the sub header is a hash of all the other
encrypted sub headers and the random data they generate as the message
travels through the network. 

Formally:

Inputs: A_1 .. A_N (addresses of intermediate nodes), 
	PK_1 .. PK_N (Public keys of intermediate nodes),
	SK_1 .. SK_N (Secret keys to be shared with intermediate nodes),
	J (Random padding)
	Type of header (END, SWAP, RTRN)
Output: H (The header)

Process: 
  // Calculate the sizes of the sub headers
  for i = 1 .. N
	SIZE_i = Len(ESHS(PK_i, V, SK_i, Z(16), Z(1), A_i))

  // Calculate the Junk that will be appended during processing:
  J_0 = ``'';
  for i = 1 .. N
	J_i = ( J_(i-1) | PRNG(HASH(SK_i, ``RANDOM JUNK'')[0:16], SIZE_i)
	J_i = J_i XOR PRNG(HASH(SK_i, ``HEADER SECRET KEY''),
  		Len(128*16))[128*16 - Len(J_i):Len(J-i)];
  end

  // Create the Header
  H_(N+1) = J[0: 128*16 - sum(SIZE_1 .. SIZE_N)];
  for i = N .. 1
	K = HASH(SK_i, ``HEADER SECRET KEY'')[0:16];
  	DIGEST = HASH( PRNG(K,Len(H_(N+1)) XOR H_(N+1) | J_i)
	if i = N then (set appropriate flag END, SWAP, or RTRN)
	H_i = ESHS(PK_i, V, SK_i, DIGEST, F, A_i) | PRNG(K,Len(H_(N+1))) XOR H_(N+1)
  end

return H_1;

Note that in order to create a reply block the last address needs to
be of type RTRN, the address before the last needs to point to the
address of the recipient and the secret keys have to be generated
from a master secret in the RTRN address block. The header is
otherwise encoded in the same way.

The Payload of messages:

The payload of a Type III remailer message has a fixed length of 32 kb
- 2*16*128 bytes = 28kb. Some information needs to be contained at the
beginning so that it can be redirected to the appropriate module:

Payload structure:

Size:   2 bytes
Module: 4 bytes
Data:   (Variable)

We should denote a payload as P.

Constructing whole messages

Given two headers and a payload one can construct a whole
message. The first header (main) must always contain a last sub header
with the flag SWAP on. The second header should contain as the last
sub header the RTRN or END flag.

Input: H1 (header containing keys H1K_1 .. H1K_N)
       and H2 (if of type END then also H2K_1 .. H2K_N)
       P (Payload)
Output: M (the message)

Process:
	// Phase 1
	if (H2 is Type END)
		for i = N .. 0
			P = BEAR(HASH(H2K_i, ``PAYLOAD ENCRYPT''),P)
		end

	// Phase 2
	H2 = BEAR(HASH(P, ``HIDE HEADER''), H2)

	for i = N .. 1
		H2 = BEAR(HASH(H1K_i, ``HEADER ENCRYPT''),H2)
		P = BEAR(HASH(H1K_i, ``PAYLOAD ENCRYPT''),P)
	end

Processing of Messages

Messages are transfered from node to node using the custom Type III
transport protocol or email. A node with private key PK receiving
message M = (H1, H2, P) performs the following operations:

Process M:
	SHS(V, SK, D, F, A) = RSA_OAEP_Decrypt(PK,H1[0:128]);
	If there is any problem with the OAEP padding discard the message.
	Check that D = H(H1[128:15*128]);
	H1 = H1[128:15*128] | PRNG(HASH(SK, ``RANDOM
						JUNK'')[0:16])[0:128];
	H1 = H1 XOR PRNG(HASH(SK, ``HEADER SECRET KEY''), Len(H1));
	H2 = BEAR(HASH(SK, ``HEADER ENCRYPT''),H2);
	P = BEAR(HASH(SK, ``PAYLOAD ENCRYPT''),P);

	if SWAP is set in F:
		H2 = BEAR(HASH(P, ``HIDE HEADER''), H2)
		Swap H1 and H2;

	if END is set in F:
		Give (A, H(SK,``APPLICATION KEY''), (H1, H2, P)) to
		Module manager. 
	else
	   	Put (H1, H2, P) in queue to be sent to A.

Single Use Reply Block exchange formats

[It should contain, use-by date, first hop address, 
 SURB block, public key, etc.
 We can also have a short hand notation for SURBS that only
 use part of the header. One only needs to specify the 
 blocks containing the SURB and a key to pad the junk]

Email Transport exchange format

[Need to specify how we are going to wrap the message
 as an email. This is only intended for SURB use and
 we should think more carefully about providing a 
 general SMTP/IMAP service.
 There is a need to provide enough information to be 
 able to block the service, for the recipient and to
 filter out messages arriving from the mix network.]

Type III remailer forward secure protocol

[cut + paste from other document]

A special channel should be established between mixes that provides
forward secrecy making it impossible to recognize or decrypt any
message that went through it in the past. In order to establish this
channel one of the two mixes initiates the connection but at the end
of the key exchange protocol the channel is bi-directional. The
protocol should be used when the TCP/IP4 address type is specified in
a sub header.

SSL does not need to be modified in any way to provide forward secure
communications. We will use a mix of ephemeral Diffie Hellman (DH)
keys and key updating to achieve our protocol goals.

Outline for SSL based protocol:

\begin{verbatim}
- A invents a new Diffie Hellman key 
  (of at least 1024 bits modulus)
  and makes a certificate signed by her signing key.
  A then initiates the SSL Handshake protocol with B.
- B invents a DH key and makes a certificate using his signing
  key.
- A can check that the Hash of the signing key is the same as
  the one contained in the Address of the sub header.
- The SSL handshake protocol proceeds as normal until a session
  key has been established. All communications are then encrypted
  using this session key.

- A sends "SEND", M, H(M,"SEND") (4 + 32k + 20 bytes)
- B sends "RECEIVED", H(M,"RECEIVED") (8 + 20 bytes)

- A sends an SSL handshake renegotiation message.
  (and MUST not reuse the same key for 
   transfering another message)
  This updates the session key and overrides the old ones.
\end{verbatim}

\emph{Note:}

The old keys must be permanently overwritten. Special care should be
taken to permanently erase them from the Hard Disk and memory. 

Only secure cipher suites should be accepted by any of the
communicating parties. Diffie Hellman is used for the key exchange
since it allows for quick key generation. The blowfish or 3DES cipher
MAY be used fas the symmetric cipher.

The standard transport mechanism over which the MixMinion Transfer
Protocol is TCP over IP. The standard listening TCP port should be 
number 48099 (until we register a port with www.iana.org)

[Q Should we request a system (<1023) or user port -GD]
[Q System ports can only be opened by root. we should avoid needing
to be root. -RD]
[Q I imagine it's hard to register a port with iana. Let's wait on
that til everybody takes us seriously. -RD]

All possible checks should be performed during the transfer protocol
and if any fail the connection MUST stop and all state MUST
be deleted. An error MAY be logged. In particular, if the address
hash element in the Master Header is nonzero, the certificate of
the communication partners must be signed using a key that hashes
appropriately.

Remailer Information Exchange format

[Need to provide signature + encryption keys, expiry dates, root of
list of hashes, Address, services and modules, policies, ...]

Statistics Information Exchange formats

[Stats, reputation, ... Roger?]
