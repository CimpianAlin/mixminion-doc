\title{Type III (Mixminion) MIX Protocol Specifications}

\section{UNRESOLVED ISSUES}

[All of these are mentioned in more detail below.]

1. Mail gateways. We should specify these.
   [Should go into appendix]
     
2. Need to write: algorithm for processing a reply.

   XXXX The thing is done, Nick please check it for bugs, and to find
   out if it is realistic. I still find it difficult to define the
   difference between forward path and SURBed messages since we do not
   have any special markers in the payload. -GD

   XXXX It looks okay; I'll triple-check it when I get there in the
   implementation (should be within the next week).  I think that I
   may be coming around to your pt of view about encoding size and
   whatnot; maybe we should mark reply deliver too?  (We need to
   figure out whether reply/junk indistinguishability really buys
   us anything.  If not, we can put more stuff in reply tags.  We
   probably need to anyway.)  If we do this, we'll want to look at all
   uses of the 'TAG' field and maybe break it up a bit finder.  -NM

3. We should write the nymserver spec too. We can keep it pretty much
    separate from this Mixminion spec.

    I will start working on this as soon as I am back from Belgium (5
    Aug 02). I will try to put on paper the scribbles of the CFP
    napkins and additional issues. May be this is a better place to
    define general SMTP transport (except for last hop of SURB)
    instead of the general mixminion spec document. -GD
    Cool. -NM    

4. Description of mixing algorithm should go in descriptor blocks. -NM

5. We must change the crossover and message-generation algorithms to address
   George's attack of 15 August 2002.

   I've taken a rough cut at this, but I want George to check it out. -NM

6. We should specify: are 'DROP'-type messages dropped before they go
    into the mix pool, or after they're pulled from the pool?

7. We should specify: what happens when a message is undeliverable?

8. Specification for incoming SMTP interface.


\section{FUTURE ISSUES}
(These are unresolved issues that we don't want to think about till we
have more stuff done.)

1. Statistics Information Exchange format
   [Not for first cut]
2. Specify: verification for directories.
	[not for first cut]
3. When do dummy messages get generated?
4. When does link padding get generated?
   [Both active research areas; not for first cut]

\section{Message Format}

\subsection{Overview}

Type III (Mixminion) MIX messages are composed of a header section and a
payload.  The header section has a main header and a
secondary header, both of which have identical structure.  Each
header is further composed of up to 16 subheaders, which are
addressed and encrypted to the intermediate nodes (MIXes).  We
begin by explaining how the full message is structured but starting
with the smallest building block.

\subsection{Definitions and cryptographic primitives}

- if B is a byte array, B[i:j] (j bytes) is sub array starting at 
  byte i with length j.
- Rand(n) (n bytes) Generates n random bytes by any secure method.
- Z(n) (n bytes) Generates n zero bytes.
- Len(M) (2 bytes) is the length of message M (* bytes).
- x|y (Len(x)+Len(y) bytes) denotes x concatenated with y.

- PAD(M,L)=M|Z(L-Len(M)) (L bytes) pads the message M (Len(M) <= L)
  to length L using zeroes.
- HASH(M) (20 bytes) is the SHA-1 hash of M (* bytes).
- PK_Encrypt(K,M) (128 bytes) The RSA-encryption of a header M 
  using the public key K.  M is padded using RSA-OAEP, and encoded
  with PKCS1.
- PK_Decrypt(K,M) (up to 86 bytes) Gives the decryption of the
  message M (128 bytes) under the private key corresponding to K.
- Encrypt(K,M) (Len(M) bytes) Rijndael encryption (in Counter mode,
  with 128-bit blocksize) of message M using key K.  (All Rijndael
  operations use 128-bit blocks.)
- Decrypt(K,M,i,j) (j-i bytes) Rijndael counter mode decryption 
  using the key material byte i to j.
- PRNG(K, n) (n bytes) Uses Rijndael in counter mode to produce N
  bytes of pseudo-random numbers.
  PRNG(K, n) = Encrypt(K, Z(n))
- SPRP_ENCRYPT(K1,K2,K3,K4,M) (Len(M) bytes) Uses a super-pseudorandom
  permutation to encrypt M with keys K1-K4.  Specifically, we use LIONESS,
  as described in XXXXCITE, with PRNG(K,n) as our stream generator,
  and the keyed-SHA1 construction specified in the LIONESS paper.

  [XXXX I think we can get away with BEAR instead.  See my email of
    September 6. -NM]

  K1 through K4 are 160 bits long.

  Thus, SPRP_ENCRYPT(K1,K2,K3,K4,M) is computed as follows:
            L := M[0:20]
            R := M[20:len(M)-20]
            R := ENCRYPT( HASH(K1 | L | K1)[0:16], R)
            L := L xor HASH(K2 | R | K2)
            R := ENCRYPT( HASH(K3 | L | K3)[0:16], R)
            L := L xor HASH(K4 | R | K4) 
            return L | R

  For convenience, we write SPRP_ENC(SK,P,M) to denote:
       SPRP_ENCRYPT(K1,K2,K3,K4,M)
       where K=HASH(SK | P)
             K1 = K
             K2 = K xor 0x00...01
             K3 = K xor 0x00...02
             K4 = K xor 0x00...03

- SPRP_DECRYPT(K1,K2,K3,K4,M) (Len(M) bytes) Inverts SPRP_ENCRYPT.

  We also define SPRP_DEC(K,P,M) as the inverse of SPRP_ENC.

RSA encryption and decryption is used with OAEP padding, using the
mask function MGF1 and hash function SHA1.  The security parameter (P
in the OAEP spec) is set to be the hash of the following 84-character
ASCII string (a quotation from Thomas Paine):

     "He who would make his own liberty secure, must guard even his
      enemy from oppression." 

(Though weaknesses have been found in OAEP's original security proofs,
they seem not to appear when you're using RSA.)

All fields are packed in Internet (MSB first) order.

All RSA encryption uses the public exponent 65537.

\subsection{The subheader structure and address extensions}

A subheader contains all the information that a MIX needs to check the
integrity of a message and route it through the Internet. The subheader
is encrypted using RSA after having been padded using OAEP, using a 1024
bit key. This results in an encrypted block of 128 bytes.

A subheader contains the following fields:

Subheader fields:

V   Version Major:   1 byte
V   Version Minor:   1 byte
SK  Shared Secret:   16 bytes
D   Digest:          20 bytes
RS  Routing Size:    2 bytes 
RT  Routing Type:    2 bytes [total 42 bytes]
RI  Routing Info:    [Routing Size] bytes

* The Version is used to manage concurrent versions of the
protocol. If a packet is received with a version that is not supported
it must be discarded. Nodes must advertise in their status blocks what
versions of the protocol they support; see below.

* The Shared Secret is the base secret that is used to generate all
other keys for the operations the node performs on the packet. It must be
kept secret and discarded as soon as the packet has been processed. 

* The Digest contains an integrity check of the remainder of the current
header (128*15 bytes in total). The digest does not cover the current
subheader: modifications to it are detected because of the OAEP padding.

* The Routing Type defines how the MIX should deliver or relay the
  message. If a MIX receives a routing type it does not recognize,
  it must discard the message.

  Most routing methods require additional addressing information.
  The Routing Size field indicates the total size of the Routing
  Information. If the information is too long to fit in a single
  subheader (more than 86-42=44 bytes), then one or more additional
  Routing Extension blocks have to be added. These additional blocks
  must be 128 bytes each and should have the following structure:
 
  Routing Extension:

    Address Data:     Variable
    Padding:          Variable

* The address data length is specified by the ``Routing Size'' field
  contained in the subheader.
* The final Routing Extension block is padded with zeroes so it is
  exactly 128 bytes.

The Routing Extension(s) corresponding to a particular subheader are
appended to the subheader, and encrypted along with the rest of the
subheaders.

We will formally refer to the subheader structure as:
SHS(V, SK, D, RS, RT, RI)     [MIN(86, 42+Len(RI)) bytes] 
And to the RSA-OAEP encrypted portions of the subheader structure as:
ESHS(PK, V, SK, D, RS, RT, RI)   [128 bytes]
And to the extension blocks for a given subheader as:
EXT(RI)                       [Ceil((Len(RI)-44)/128) * 128 bytes]

\subsection{Routing information}

There are 5 predefined routing types:

0x0000-0x00FF: PROTOCOL SUPPORT

0x0000 DROP    (0 bytes of routing information)
0x0001 FWD/IP4 (IP: 4 bytes, PORT: 2 bytes, KEYID: 20 bytes): 26 bytes
0x0002 SWAP-FWD/IPV4 (same info as FWD/IP4)

0x0100-0x0FFF: PREDEFINED DELIVERY TYPES.

0x0100 SMTP   (EMAIL ADDRESS: variable, TAG: variable) Variable bytes
0x0101 MBOX   (USER: variable, TAG: variable) Variable bytes
0x0102 MIX2   (EMAIL ADDRESS: variable).  Type-2 remailer support.

0x1000-0xEFFF: UNALLOCATED

0xF000-0xFFFF: FOR EXPERIMENTAL USE

A DROP routing type indicates a dummy message. It must be discarded.

A FWD/IP4 routing type indicates that the message must be
retransmitted using the TLS/Mixmaster transport protocol. The IP field
represents the IPv4 address.  The KEYID field contains the SHA1 hash
of the ASN.1 representation of the next node's transport public key.
(Note that a node's transport key does not need to be the same as the
key it uses to decrypt subheaders.)

A SWAP routing type tells the node to exchange headers as described below.

See appendices for more information about SMTP and MBOX delivery.

\subsection{The header structure}

Each type III message has two headers with identical structure. These
headers are swapped at the crossover point.

A header is 16*128 bytes long and contains up to 16
subheaders. Starting with N subheaders SH_0..SH_N containing secrets
SK_0..SK_N (and placing routing extension blocks directly after their
respective subheaders), the header is constructed by appending 
random padding to achieve a total size
of 128*16 bytes. Then, each subheader key is used to create a key
Hash(SharedSecret | ``HEADER SECRET KEY'') with which the part of the
header after the subheader (but including its routing extension) is
encrypted using counter-mode AES.

We construct the subheaders from last to first, so that each can contain
a digest of the subsequent subheaders and padding data.

PROCEDURE: Create a single header.

Inputs: A_1 .. A_N (addresses of intermediate nodes), 
	PK_1 .. PK_N (Public keys of intermediate nodes),
	SK_1 .. SK_N (Secret keys to be shared with intermediate nodes),
        R Routing type and information of last header (FWD, DROP, SWAP, etc.)
Output: H (The header)

Process: 
  // Calculate the sizes of the subheaders
  for i = 1 .. N
	SIZE_i = 128 + Len(EXT(RI_i))

  // Calculate the Junk that will be appended during processing:
  J_0 = ``'';
  for i = 1 .. N
	J_i = J_(i-1) | PRNG(HASH(SK_i, ``RANDOM JUNK'')[0:16], SIZE_i)

        Stream_i = PRNG(HASH(SK_i, ``HEADER SECRET KEY''), 128*16);

	J_i = J_i XOR Stream_i[128*15 -Len(J_i) + SIZE_i:Len(J_i)];
  end

  // Create the Header
  H_(N+1) = Rand(128*16 - sum(SIZE_1 .. SIZE_N));

  for i = N .. 1
	K = HASH(SK_i | ``HEADER SECRET KEY'')[0:16];
	IF i = N (set appropriate routing type and A_i)
	EH = EXT( RI_i )
        REST = Encrypt(K, (EH | H_(i+1)))
  	DIGEST = HASH(REST | J_i)
	H_i = ESHS(PK_i, V, SK_i, DIGEST, F, len(RI_i), RT_i, RI_i) | REST
  end

return H_1;

\subsection{The Payload of messages}

The payload of a Mixminion message has a fixed length of 32 kb
- 2*16*128 bytes = 28kb.   Payloads indicate their size.

(When sending a reply message with a SURB, we use payload encryption
to prevent the crossover point from seeing an unencrypted payload. See
'SURB binary format' for more information.)

We denote a payload as P.

\subsection{Constructing messages}

Given two headers and a payload one can construct a
message. The first header must contain a subheader
with routing type SWAP.  

PROCEDURE: Construct a message.

Input: H1 (header containing keys SK1_1... SK1_N)
       and H2 (either a header containing keys SK2_1... SK2_N if
         we constructed it, or a header with unknown keys if we're
         using a reply block and a SURB secret key.)
       P (Payload)
Output: M (the message)

Process:
        // Phase 1
        if (H2 is a reply block)
                P = SPRP_ENC(SURB secret key, P)
	else // (H2 is *not* a reply block)
		for i = N .. 1
	            P = SPRP_ENC(SK2_i, "PAYLOAD ENCRYPT", P)
		end
        else
	// Phase 2
	H2 = SPRP_ENC(SHA1(P), ``HIDE HEADER'', H2)
[XXXX We should add this to address George's attack of 15Aug.  George,
      is this correct?  Does it go here?
 XXXX This is correct -GD
        P = SPRP_ENC(SHA1(H2), "HIDE PAYLOAD", P)
                                                       - NM]

	for i = N .. 1
		H2 = SPRP_ENC(SK1_i, "HEADER ENCRYPT",H2)
		P = SPRP_ENC(SK1_i, "PAYLOAD ENCRYPT",P)
	end
	M = (H1, H2, P)

\section{Processing of Messages}

Messages are transferred from node to node using either the custom Type
III transport protocol (see below) or email.  A node with private key
PK receiving message M = (H1, H2, P) performs the following operations:

PROCEDURE: Process a message M
	SHS(V, SK, D, RS, RT, RI) = PK_Decrypt(PK,H1[0:128]);
        If there is any problem with the OAEP padding discard the message.
        Check that D = HASH(H1[128:15*128]), and discard if not.
        Let n_extra = number of extended headers = Ceil( (RS-44) / 128 )
                  
        H1 = H1[128:15*128] | PRNG(HASH(SK | "RANDOM 
                                               JUNK")[0:16],128+128*n_extra)
	H1 = H1 XOR PRNG(HASH(SK, "HEADER SECRET KEY")[0:16], Len(H1))
        RI = RI | H[0:128*n_extra]
        H1 = H1[128*n_extra:128*16]
	H2 = SPRP_DEC(SK, ``HEADER ENCRYPT'',H2);
	P = SPRP_DEC(SK, ``PAYLOAD ENCRYPT'',P);

	if routing type is DROP:
                End.
	if routing type is SWAP-FWD:
[XXXX We should add this to address George's attack of 15Aug.  George,
      is this correct?  Does it go here?
[XXXX I think this is correct -GD]
                P = SPRP_DEC(SHA1(H2), "HIDE PAYLOAD", P)
                                                             -NM]
		H2 = SPRP_DEC(SHA1(P), ``HIDE HEADER'', H2)
		Swap H1 and H2;
        if routing type is SWAP-FWD or FWD:
	   	Put (H1, H2, P) in queue to be sent to the address in RI.
        Otherwise:
		Give (RT, RI, HASH(SK,``APPLICATION KEY''), P) to
Module manager. 

\section{Decoding of messages}

Messages that are received by a client can either be sent using the
forward path, or a SURB. They might either arrive in a mixminion
format, that includes all the headers, or stripped of the two headers
with only the a TAG field attached to them.

A client that receives a message that is ultimately destined to them
should perform the following operations to decode it:

PROCEDURE: Decode a message.

Input:  TAG field of sub-header or header where 
        TAG = ( Encrypt(KEY, nHops | seed) | padding up to 44b).
        M the body of the message.
Output: P, the plaintext of the message.

	If the message was sent using the forward-path then 
		P = M; exit;
	Otherwise, we regenerate all keys used to encrypt the payload:
		(seed, nHops) = Decrypt(KEY,TAG)[0:17]
		SK_1..SK_16 = PRNG(seed, nHops*16)[0:16*nhops]
		For i = nHops to 0
			M = SPRP_DEC(SK_i, ``PAYLOAD ENCRYPT'',M);
		end

		// We need here a convention for creating the
		//   Encryption key in the SURB.
		KEYX = HASH(seed| ``PRIVATE SURB KEY'')[0:16];
		M = SPRP_DEC(KEYX,M);
		
		P = M; exit;

[I am a bit uneasy that we have not defined any redundancy, or marker,
that would allow the final recipient to decide if this is a forward
path message or a SURBed message. Also the size of valid bytes has
been scrapped, which does not help in extracting the valid bits from
the junk. -GD]

\section{Single Use Reply Block exchange formats}

A SURB can be encoded in a standard binary or ASCII format.

Binary Format:

   Begin Marker: 4 bytes
   Version:      2 bytes
   Use-by-Date:  4 bytes
   SURB header:  2048 bytes
   Routing Size: 2 bytes
   Routing Type: 2 bytes
   Encryption key: 16 bytes
   Routing Info: (Routing Size) bytes

   Total: 14 bytes + Header size + Routing info size.

* The begin marker is the ASCII 4-byte string 'SURB'. 
* The version number contains the format version of the SURB.
  (should be hex 01 and 00 for this standard).
* Routing type/routing size/routing info: Definded as in subheaders.  
  These fields encode the address of the hop which the SURB user should
  use as an exit point.
* Use-by-Date: indicated the expiry date the SURB should be used by. Can
  be calculated using the key rotation frequencies of the intermediate
  nodes.  This field must be given as a number of seconds since
  midnight GMT on Jan 1, 1970 -- but must be aligned to the start of a
  day (in other words, it must be divisible by 60*60*24).
  (Misaligned dates must be rejected as invalid.)

  (Rationale: a seconds-level granularity allows us to move to a
  tighter schedule later on in order to support a synchronous mixnet.)

* SURB data: Containst the SURB that is created as described
  above. 
* Encryption key: used to LIONESS-encrypt the payload before sending it
  into the network.  

  [XXXX This prevents the crossover point from seeing an
    unencrypted payload.  However, using symmetric crypto requires the
    SURB generator to keep SURBs confidential from everyone but their
    users.  George has suggested that we use PK instead, but generating
    a fresh RSA key for each SURB slows SURB generation down by a factor
    of 30-70.   Perhaps a two-mode system is in order.  Hmm. -NM]
  [XXXX A bit more discussion on this topic: In fact there is, from an
    anonymity point of view, very little to gain from using asymetric
    crypto. The original point I had in mind had to do with
    observability. If the key is symetric then crooked nodes can do
    the following:
	- observe all users noting which SURB's they are receiving.
	  They can then note down the keys used.
	- When a message is seen at a cross over point, all keys are
    	  tried on it, making the forward path useless, since it can
          be linked to a SURB received by someone specific.

    My original idea was to use a key-private cryptosystem, to hide
    which key was used to encrypt the payload.
    I just realised that the above is rubish. If the SURB is not
    delivered to receipient anonymously and privately, then it cannot
    be used by the recipient anonymously, since it appears as it is in
    the header of the packet at the crossover point. Therefore
    symetric cryptography is as good as asymetric, key-privat in that
    scenario. But this is a point we need to emphasise, that is
    obvious when mentioned:

	``Do not use SURB's that have been delivered non anonymously
	for establishing sender-anonymous communications with others.''
								-GD]

The ASCII Encoding of SURBs.

The  ASCII compatible format of SURBs is:
-----BEGIN SURB-----
Version: x.x
Base64 encoded binary SURB 
-----END SURB-----

The version number should be in decimal ASCII and is the same as the
binary version.

\subsection{'Stateless' reply blocks}

4. Stateless replies and SMTP (depends on 2 and 3, if I understand correctly)

If a client does not wish to remember all of her outstanding
reply blocks, she may generate them in 'stateless' mode.  She  
does so by using an SMTP or MBOX delivery type, and setting
the TAG field to 

           ( Encrypt(KEY, nHops | seed) | padding up to 44b)
           [nHops: 1 byte; seed: 16 bytes.]

She uses PRNG(seed, nHops*16) to form the up-to-16 SK's for the reply.
She also uses the seed to create the symetric key present in the SURB
as KEYX = HASH(seed| ``PRIVATE SURB KEY'')[0:16];

To understand a message later, the client need only remember (or be
able to reconstruct) KEY.  

[Note 1: It would be best for deniability if KEY were the SHA1 hash of
some secure password.  On the other hand, since an adversary could
then mount an off-line passing attack on KEY, and since most people
can't construct or remember a good password, it would probably be
safest to store KEY on disk, password-encrypted.  All implementations
of stateless replies must support at least this latter mode.]

[Note 2: 'Encrypt' here is not AES in counter mode; that would be
madness.  Instead, we use AES in CBC mode.]
[ XXXX if we use CBC mode we will need to use a different IV everytime -GD]

\section{Replay Avoidance}

The nodes MUST implement a mechanism to make sure that messages cannot
be replayed. To do this a hash of the secret contained in the
subheader is kept for as long as the public key under which it was
encrypted is in use. The Hash should be computed in the following way:

X = HASH(SharedSecret | ``REPLAY PREVENTION'')

The value X is not secret, and its secrecy should not be relied upon.
The integrity of the list should be secured and the X values lists may
be made public.

\section{Type III (Mixminion) forward secure protocol}

A special channel should be established between mixes that provides
forward secrecy making it impossible to recognize or decrypt any
message that went through it in the past. In order to establish this
channel one of the two mixes initiates the connection but at the end
of the key exchange protocol the channel is bi-directional. The
protocol should be used when the SWAP-FWD/IP4 or FWD/IP4 address type
is specified in a subheader.

The Mixminion protocol uses TLS (the IETF standardization of SSL) with
the ciphersuite "TLS_DHE_RSA_WITH_AES_128_CBC_SHA" (defined in
tls-ciphersuite-03.txt).  No other ciphersuite is permitted for
MIX-to-MIX communications.

[Servers must allow incoming connections via SSL3_RSA_DES_192_CBC3_SHA
for clients written with older SSL libraries.  However, servers must
never initiate connections with this suite.]

X.509 certificates need not be signed; instead, they must contain
a key matching that used in the KEYIDportion of the header's routing
data.  

Messages are sent from client to server.  Session suspension is
permitted; however, the client must send a ClientHello packet to
renegotiate session key whenever it is done sending a batch of
messages.  The server should allow session resumption later only if 1)
less than 120 seconds have passed, or 2) the client re-keyed
immediately before suspending.  [This way, no key that has been used
to send messages can be used after 120 seconds to send messages
again. -NM]

Protocol outline: (Portions marked with '*' are normative; other
portions are non-normative descriptions of TLS.)

\begin{verbatim}
- A invents a new Diffie Hellman key 
  (of at least 1024 bits modulus)
  and makes a certificate signed by her signing key.
  A then initiates the SSL Handshake protocol with B.
- B invents a DH key and makes a certificate using his signing
  key.
* A checks that the Hash of the signing key is the same as
  the one contained in the routing info of the subheader.
- The SSL handshake protocol proceeds as normal until a session
  key has been established. All communications are then encrypted
  using this session key.

* A sends "MMTP 1.0", CRLF.  This indicates the protocol versions that
  A supports.

  (Future clients that support more protocols should transmit
   "MMTP", a list of comma-separated protocol versions, and a CRLF.)

* If B is not willing to use any protocol A supports, B closes the 
  connection.

  B sends "MMTP 1.0", CRLF.  This indicates B's choice of protocol.

  If A is not willing to support B's choice, A closes the connection.

* Message case:

     * A sends "SEND", CRLF, M, HASH(M|"SEND") (6 + 32k + 20 bytes)
     * B sends "RECEIVED", CRLF, HASH(M|"RECEIVED") (10 + 20 bytes)

* Padding case:

     * A sends "JUNK", CRLF, Junk, HASH(M|"JUNK") (6 + 32k + 20 bytes)
       (where Junk is an arbitrary 32k sequence."
     * B sends "RECEIVED", CRLF, HASH(M|"RECEIVED JUNK") (10 +20 bytes)

       [Note that both cases require the same number of bytes and 
        processing time.]

* After sending a batch of messages, A sends an TLS handshake
  renegotiation message. This updates the session key and
  overrides the old ones.

\end{verbatim}

\emph{Note:}

The old keys must be permanently overwritten. Special care should be
taken to permanently erase them from the Hard Disk and memory. 

The standard transport mechanism over which the MixMinion Transfer
Protocol talks is TCP over IP. The standard listening TCP port should be 
number 48099 (until we register a port with www.iana.org)

All possible checks should be performed during the transfer protocol
and if any fail the connection MUST stop and all state MUST
be deleted. An error MAY be logged. In particular, if the address
hash element in the Master Header is nonzero, the certificate of
the communication partners must be signed using a key that hashes
appropriately.

\section{MIX Information Exchange format}

In order to automate and standardize directory servers, we provide 
a standardized extensible server descriptor format.

All server descriptors and statistics blocks follow a simple
section-based key/value format, with items loosely based on RFC822.

[Section1]
Key: Value
Key: Value
Key: Value

Key: Value
Key: Value

[Section2]
Key: Value

\subsection{Syntax}

(Notation:  X*: 0 or more occurences of X.
            X+: 1 or more occurences of X.
	    X?: 0 or 1 occurrences of X.
            X Y: An occurrence of X followed by an occurrence of Y.
	    X*{Y}: 0 or more occurrences of X separated by occurences
                  of Y.
            X|Y: Either an occurrence of X, or an occurence of Y.)

Descriptor = CRLF* Section+ 

Doctype = (<any printable character but '-'>)+

Section = SectionLine EntryLine*

SectionLine = '[' Word ']' CRLF+

EntryLine = Word ':' ' ' Data CRLF+

Word = (<Any printable, non-space character but ':'>)+

Data = (<any character but CR or LF>)*

CRLF = CR LF

\section{Mixminion descriptor blocks}

This section describes the format of server descriptors, as uploaded
to and downloaded from directory servers.  A server descriptor is a
promise, by a MIX's administrators, to provide a given set of
services, keys, and exit policies over a set period of time.

The first section must be a 'Server' section.  This section includes
the entries:

     'Descriptor-Version':  the string "1.0"
     'IP': An IPv4 address, in dotted-quad format.
     'Nickname': A human-readable identifier for this server.  If it
         contains any periods, it must be a fully qualified DNS name
         which resolves to the provided IP for the entire lifetime of
         this Descriptor block.  It must be no more than 128 characters.
     'Identity': The modulus of this Mix node's long-term signing key,
         represented in ASN.1, and encoded in BASE64.  Whitespace in
         this field is ignored, to allow the key to span multiple
         lines.  The modulus of this key should be at least 2048 bits
         long and no more than 4096 bits long.  The exponent of this 
         key must be 65537.

	 Clients should at least give a warning if the identity key of
         any server should ever change.
     'Digest': The digest of this block. See below.
     'Signature': The signed digest of this block.  See below.
     'Published': A date/time, in the form 'DD/MM/YYYY HH:MM:SS',
         for when this block was generated.
     'Valid-After': A date, in the form 'DD/MM/YYYY'.  After midnight GMT
         on this date, this server must support the operations listed
         in this descriptor.
     'Valid-Until': A date, in the form 'DD/MM/YYYY'.  Until midnight
         GMT on this date, this server must support the operations listed
         in this descriptor.
     'Contact': An email address that may be used to contact the
         administrator of this server. Optional field.  Must be no
         more than 256 characters.
     'Comments': Human-readable information about this server.  Must
         be <1024 bytes long.  It *must not* be necessary to read this
         information to use the server properly.
     'Packet-Key': The public key used to encode encode subheaders for
         this server, encoded in ASN.1, represented in BASE64. 

The digest of a descriptor block is computed by removing the contents
of the digest and signature fields, and computing the SHA-1 digest of
the result.  (That is, ``Digest: DATADATADATA...'' is replaced with
``Digest:''.)  The signed digest is the OAEP/PCKS1 signature of the
digest with the server's identity key.  This value is represented in
BASE64.

If this server accepts incoming MMTP connections, it MAY have an
'Incoming/MMTP' section, with the following entries:

     'Version': The string '1.0'
     'Port': A port at which IP accepts incoming MMTP connections.
     'Key-Digest': The KEYID of this server, encoded in BASE64.
     'Protocols': A comma-separated list of the protocols this
           server accepts.

and any number of entries of the form:
     'Allow': Address Pattern
     'Deny': Address Pattern

If this server supports outgoing MMTP connections, it MAY have a
'Outgoing/MMTP' section, with one entry each of the form:

      'Version': The string '1.0'
      'Protocols': A comma-separated list of the protocols this server
           supports for outgoing connections.

and any number of entries of the form:

      'Allow': Address Pattern
      'Deny': Address Pattern

The Address Pattern tokens are of the form:

   AddressPattern = (IP ('/' Mask)? | '*') (Port ('-' MaxPort)?)?

'*' is a synonym for '0.0.0.0/0.0.0.0'.

An omitted mask defaults to 255.255.255.255.  An omitted portrange
defaults to 48099 on ALLOW and 0-65535 on DENY.

The entries are order-significant; the first one to match wins.

The default policy is 'Deny: *'

If this server supports outgoing delivery for a module ABCD, it will
have a [Delivery/ABCD] section.  See appendices for more detail on
specific modules, including SMTP and MBOX.

Other services provided by this server should each have their own section.

(Note: A server need not advertise all of its capabilities; it is
permissible (for example) for a server that supports incoming MMTP
connections to omit the Incoming/MMTP section.)

A client should ignore any sections it does not recognize, but should
not use any service whose sections have an unrecognized descriptor
version.

\subsection{Directories and Directory servers}

A directory is a list of Mixminion servers which are believed to
be operational at a given time.

A directory server provides an HTTP URL for uploading server
descriptors, an HTTP URL for downloading a directory, and a long-term
public key (2048-4096 bits).

To upload a descriptor block, a client performs an HTTP POST request
to the upload URL, with the server block as enclosed entity.

To retrieve the directory, a client performs an HTTP GET request on
the directory URL.

A directory takes the following form:

<mixminion-directory>
  <version>1.0</version>
  <identity>Base64-encoded public key, in ASN.1</identity>
  <signature>Base64-encoded OAEP/PCKS1 signature of this document, with
     the contents of this field removed.</signature>
  <server>
     (Server descriptor block)
  </server>
  <server>
     (Server descriptor block)
  </server>
   .....
</mixminion-directory>

Directory servers change their directories only at midnight GMT.  Any
client which has not downloaded a directory since before midnight GMT,
must download a fresh directory before generating any packets.

A directory includes all the servers that were uploaded to the
directory before some cutoff time the previous day, and which proved
upon some random number of tests and probings to have a real Mixminion
server running on them.  A directory server periodically re-tests
the servers in its directory to make sure they have not gone down.

Because of possible partitioning attacks related to accidentally or
maliciously unsynchronized servers, the presence of multiple directory
servers presents sever security issues.  Since solving these issues is
an active research project, we leave them for a later draft.

[XXXX Issues include:  How do directory servers synchronize?
   What happens when they disagree?  How many servers must a client
   contact before he/she has enough information?  How do we catch
   dishonest directory servers? -NM

\section{Appendix: Pooling rule}

In order to allow room for future experimentation, we do not require a
single batching rule.  Nonetheless, we describe a recommended rule (as
used in Mixmaster) which is somewhat resistant to flooding attacks.
Implementors are strongly encouraged to use this algorithm, or another
equally robust against active and passive attacks.  (Be sure to read
\cite{batching-taxonomy}.)

PROCEDURE: Choose sets of messages to transmit ("Cotterll-style batching")

Inputs: Q (a queue of messages)
        N (the number of messages in the queue).
	MIX_INTERVAL (algorithm parameter; time to wait between
                      batches of messages.  Should be around XXXXX.)
        POOL_SIZE (algorithm parameter; minimum size of pool.  Should
                   be at least XXXXXXXX)
        MAX_REPLACEMENT_RATE (algorithm parameter; largest allowable
                   rate for messages to be removed from the
                   pool. Should be between XXXX and XXXX.)

Outputs: (A set of messages sent to the network).

1. Wait for MIX_INTERVAL seconds.

2. If N > POOL_SIZE, then choose Min(N-POOL_SIZE, N*MAX_REPLACEMENT_RATE)
   messages from Q.  Transmit the selected messages.

3. Repeat indefinitely.

[XXXX Erg.  I'm changing this back again.  If I understand correctly,
   there's a paper that describes another batching algorithm and calls
   it "mixmaster".  Andrei, Paul, and Roger copied this description, and
   I copied them. 

   Paul and/or Andrei describe a variant that checks N>=POOL_SIZE+THRESHOLD, 
   with THRESHOLD>= 1.  Roger claims (verbally) that he isn't sure whether
   this would buy us anything, since (he says) adding 1 to POOL_SIZE would
   always increase anonymity more than adding 1 to THRESHOLD.  Once they've
   come to some agreement, maybe we should do that instead.

   Then there are binomial mixes, where instead of sending b messages,
   you send each message with probability=b/POOLSIZE.  Roger/Paul/Andrei
   seem to like those, but I don't have a clear sense of how sure they are,
   and how much everyone agrees with them. -NM]
  
\section{Appendix: MBOX delivery}

Servers that want to support MBOX delivery have an internal list of
users they accept messages to, and an internal mapping from those
users to some delivery mechanism for each one.  Typically, this is a
mapping from 'username' to 'username@localhost', and delivery defaults
to local delivery via sendmail.  Servers are free to provide other
implementations for MBOX delivery.

MBOX delivery differs from SMTP delivery in that it is not intended
for addressing messages to arbitrary SMTP addresses.

Servers that support MBOX delivery MAY include a [Delivery/MBOX]
section, containing only the entry "Version: 1.0".

The MBOX routing type is used for messages to be delivered to a local
user.  The USER field must be NUL-terminated; the TAG field is
free-form. 

\section{Appendix: SMTP delivery}

At the final hop, when the delivery mechanism is SMTP, we proceed as
follows.  If the message is a series of printable characters followed
by some number of NULs, assume we're delivering in ASCII an ISO-XXXX
character set, and send the text portion of the message as an email.
(Where printable == {all characters but hexadecimal 00-06,0E-1F}).
Otherwise, ASCII-armor the message as in 'email transport exchange
format' below.

[This way, plaintext forward messages are delivered as plaintext,
and tagged messages, reply messages, and non-plaintext messages are
all delivered as junk.]

Servers supporting SMTP MAY include a [Outgoing/SMTP] section,
containing only the entry "Version: 1.0".

Servers SHOULD include a note with every SMTP, explaining that the
message is delivered anonymously, and providing an opt-out address and
an abuse contact.

The EMAIL field in the SMTP routing type should be a valid mailbox
[RFC2821]. A mailbox is the canonical form of the ``user@domain''
part of an e-mail address. Mixminion uses only mailboxes, because the
display name and comment parts of an e-mail address could potentially be
different for senders who have obtained an address from different
sources (leading to smaller anonymity sets). The EMAIL field must be
NUL-terminated.

The TAG field is appended to the message in an X-Remailer-Tag header.

[XXXX Until we have better answers about abuse prevention, nobody should
  actually implement an SMTP module. :) -NM]

\section{Appendix: Backward compatibility with type-II remailers}

In order to share anonymity sets with type-III remailers while
retaining type-II support, some remailers may wish to use Mixminion to
deliver type-II messages.  This is done as follows:

Nodes that accept both type-II and type-III messages may advertise the
fact in their server descriptor by including a section of the form:
 
         [Incoming/Mix2]
         Address: (type-II remailer's email address)
         Key: (type-II key)
	 KeyID: (type-II keyid)
         Signature: (signature of identity key with type-II key)
	 (Optionally, KeyID and Signature repeated any number of
                      times.)

This section advertises that the mix can handle type-II messages
intended for a given type-II identity (email address) and set of keys.

The value of 'key' is the base-64 representation of the ASN.1 encoding
of the Mix node's type-II key. The value of 'Signature' must be the
base-64 representation of the RSA-OAEP/PKCS1 signature (using the
type-II remailer key) of the SHA-1 hash of the ASN.1 representation of
this node's identity key.

Directory servers and bridging nodes _must_ verify that keyid and
signature are correctly computed.

Upon receiving a type-II message via SMTP, a bridging node checks
whether the destination node is also a type-III node, by looking for a
type-III node whose KeyID matches the KeyID for the packet. [See below]
If it finds one, the bridging node unbase64's the type-II message's 
contents, and uses them (plus random padding) as the payload
of a type-III message for that node.  The routing type must be 'MIX2'
(0x0102); the routing info must be equal to the destination mix's
type-II address.

\subsection{Non-normative note: extracting KeyID and message contents}

(This information is included in the Type-II remailer spec; it is included
here only for reference.)

A type-II message follows the format:

"-----BEGIN REMAILER MESSAGE-----" NL
PacketLength NL
Checksum NL
Packet NL
"-----END REMAILER MESSAGE-----" NL

PacketLength is the length of the packet in bytes, encoded as a
decimal integer."Checksum" is equal to the MD5 hash of the packet,
encoded in Base 64.  The packet is also encoded in base 64; The first
16 bytes of the packet are the KeyID of the recipient.

The KeyID of a type-II node is calculated by taking the public key
<n,e>, expressing n and e as zero-padded, big-endian integers,
concatenating them, and taking the MD5 hash of the result.

When encoding a type-II message for transmission in a type-III payload,
a type-III node should include:

S   [2 bytes] (Size, big-endian)
CHK [16 bytes] (MD5 checksum as given in type-II packet, base-64 decoded)
PKT [S bytes] (Packet as given in in type-II packet, base-64 decoded)
PAD [28KB-16-2-S bytes] (Random padding)

